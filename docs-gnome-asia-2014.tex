\documentclass{beamer}

\include{gnomeasia-theme}

\usepackage[english]{babel}
% or whatever

\usepackage[latin1]{inputenc}
% or whatever

\usepackage{times}
\usepackage[T1]{fontenc}
% Or whatever. Note that the encoding and the font should match. If T1
% does not look nice, try deleting the line with the fontenc.


\title%[Short Paper Title] (optional, use only with long paper titles)
{GNOME Documentation:\\ helping you learn and give back}

%\subtitle
%{Include Only If Paper Has a Subtitle}

\author
{Ekaterina Gerasimova, \texttt{kittykat3756@gmail.com}}
% - Give the names in the same order as the appear in the paper.
% - Use the \inst{?} command only if the authors have different
%   affiliation.

\date%[GNOME.Asia 2014] % (optional, should be abbreviation of conference name)
{GNOME.Asia Summit 2014}
% - Not really informative to the audience, more for people (including
%   yourself) who are reading the slides online

%\subject{Theoretical Computer Science}
% This is only inserted into the PDF information catalog. Can be left
% out. 

\begin{document}

\begin{frame}
  \titlepage
\end{frame}

\section{How does it work?}

\subsection{Who are the documentation team?}

\begin{frame}{Who are the documentation team?}
  \begin{itemize}
  \item
    Established around 1999
  \item
    Sun Microsystems were heavily involved in GNOME documentation in the early days
  \item
    Now made up mostly of volunteer contributors
  \item
    Valuable contributions from other community members, especially translators
  \end{itemize}
\end{frame}

\begin{frame}{}
  \pgfdeclareimage[height=5cm]{map-team}{map.png}
  \center{\pgfuseimage{map-team}}
\end{frame}

\begin{frame}{}
  \pgfdeclareimage[height=5cm]{map-contribs}{map-contribs.png}
  \center{\pgfuseimage{map-contribs}}
\end{frame}

\begin{frame}{}
  \pgfdeclareimage[height=5cm]{map-china}{map-china.png}
  \center{\pgfuseimage{map-china}}
\end{frame}

\begin{frame}{What does the documentation team do?}
  \begin{itemize}
  \item
    User help: http://help.gnome.org/user/
  \item
    System administrator guide: http://help.gnome.org/admin/
  \item
    Developer documentation: http://developer.gnome.org
  \end{itemize}
\end{frame}

\begin{frame}{What tools does the team use?}
  \begin{itemize}
  \item
    Mallard
    \begin{itemize}
      \item
        XML, similar to HTML
      \item
        designed for application and desktop documentation
    \end{itemize}
  \item
    yelp-tools
    \begin{itemize}
      \item
        set of helpful command-line tools
      \item
 	great for validating Mallard
    \end{itemize}
  \item
    text editor (any is suitable)
  \end{itemize}
\end{frame}

\begin{frame}{What about the help itself?}
  \begin{itemize}
  \item
    Task oriented: short pages explaining how to achieve one goal
  \item
    Extensible: easy to add pages one at a time
  \item
    Limited markup: around 50 elements, much fewer than alternatives
  \item
    Easy to preview on the go: run 'yelp help/C/' for most applications
  \item
    Licensed under CC-by-SA 3.0
  \end{itemize}
\end{frame}

\begin{frame}{How does the team work?}
  \begin{itemize}
  \item
    Efforts coordinated on wiki, IRC and mailing list
  \item
    At least two in-person meetups per year
  \item
    Organised online sprints around release time
  \item
    Team members "maintain" docs for individual projects
  \end{itemize}
\end{frame}

\section{How can you help?}

\subsection{Why are you here?}

\begin{frame}{Why are you here?}%{Subtitles are optional.}
  \pgfdeclareimage[height=4.5cm]{bad-flowchart}{bad-flowchart.pdf}
  \center{\pgfuseimage{bad-flowchart}}
  % If you think the first step towards contributing is sending in a patch, you're probably wrong.
\end{frame}

\begin{frame}{It's all about the people}
  \begin{itemize}
  \item
    Talk to people\ldots
    \begin{itemize}
    \item
      IRC
    \item
      Mailing lists
    \item
      Here and now
    \end{itemize}
  \item
    Use available resources
  \item
    Ask for help
  \end{itemize}
\end{frame}

\begin{frame}{Code of conduct}
  \begin{itemize}
  \item
    Assume people mean well
  \item
    Try to be concise
  \item
    Be patient and generous
  \item
    Be respectful and considerate
  \end{itemize}
\end{frame}

\begin{frame}{Workflow}
  \begin{itemize}
  \item
    Find out about the project workflow and follow it
    \begin{itemize}
    \item
      Bugzilla
    \item
      Mailing lists
    \item
      IRC
    \item
      Email
    \end{itemize}
  \item
    Build and test
  \end{itemize}
\end{frame}

\begin{frame}{How things happen}
  \pgfdeclareimage[height=4.5cm]{good-flowchart}{good-flowchart.pdf}
  \center{\pgfuseimage{good-flowchart}}
\end{frame}

\begin{frame}{Followup}
  \begin{itemize}
  \item
    Be patient
  \item
    Follow up with the reviewer
  \item
    Follow up on the review
  \end{itemize}
\end{frame}

\begin{frame}{Make good contributions}
  \begin{itemize}
  \item
    Use the reviewer's time well
  \item
    Keep to the style of the project
  \item
    Follow the review process
  \item
    Respond in a timely manner
  \item
    Write good commit messages
  \end{itemize}
\end{frame}

\begin{frame}{Experience and growth}
  \begin{itemize}
  \item
    The more you contribute, the more you communicate with people
  \item
    The more you communicate, the more you become part of the community
  \item
    Once you are part of the community, you can become a Foundation member
  \end{itemize} 
  \begin{itemize}
  \item
    Help others?
  \end{itemize}
\end{frame}

\begin{frame}{Summary}
  % Keep the summary *very short*.
  \begin{itemize}
  \item
    The \alert{passion} for a FLOSS project comes from your personal interest in the project
  \item
    \alert{Collaborate} with others and \alert{learn together} to get ahead
  \item
    \alert{Explore} other projects
  \end{itemize}
  
  With special thanks to
  \begin{itemize}
  \item
    Andr\'{e} \v{C}. Klapper, \texttt{ak-47@gmx.net} 
  \item
    Sindhu S, \texttt{sindhus@gnome.org}
  \end{itemize}
\end{frame}

% All of the following is optional and typically not needed. 
\section<presentation>*{Resources}
\subsection<presentation>*{Resources}

\begin{frame}{Resources}
    
  \begin{thebibliography}{10}

  \bibitem{code}
    Documentation team workspace:
    \newblock https://wiki.gnome.org/DocumentationProject/
  \bibitem{code}
    Source code:
    \newblock https://git.gnome.org/
  \bibitem{code}
    Bugzilla:
    \newblock https://bugzilla.gnome.org/
  \bibitem{code}
    Mallard references:
    \newblock http://projectmallard.org/\\ http://flossmanuals.net/introduction-to-mallard/

  \end{thebibliography}
\end{frame}

\end{document}
